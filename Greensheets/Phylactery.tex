\documentclass[green]{Sel}
\begin{document}
\name{\gPhylactery{}}
A human sorcerer can only transform into a lich when holding the item they have personally prepared to be their phylactery. That item immediately becomes their phylactery. No one but the sorcerer themselves can determine whether a given non-magical item has been prepared to be a phylactery. For a human sorcerer, the process of preparing a phylactery takes years of study and violence-- often a hundred murders or more. 

Once a lich, a sorcerer can switch phylacteries with relative ease, though they cannot split their power or otherwise maintain a back-up phylactery.

Phylacteries cannot be made from items that can be eaten, are able to rot, are power sources, or are already have magical properties.

Phylacteries cannot be charmed with any additional spells.

In order to cast a spell, a lich must be holding their phylactery or carrying it on their person. Liches cannot draw on energy from another lich's phylactery.

Phylacteries are represented as envelopes. The outside of the envelope will be labeled with the phylactery's physical form. Inside the envelope is a card with the owner's true name and the number of units of power contained in the phylactery written on it. For example, if a lich with the true name Servius has a phylactery made out of an old book and containing two units of power, their phylactery would be represented by an envelope labeled "old book" and containing a card saying "The owner's true name is Servius. Power level: 2."

If you touch an item, look inside the corresponding envelope for a card. If the item turns out to be a phylactery, then you automatically learn the owner's true name and current power level.

Unless you know otherwise, casting a spell costs a lich 1 unit of power. You cannot cast a spell unless you can pay the full cost. When you cast a spell, update the power level written on the card in your phylactery's envelope.

Humans are less efficient spellcasters, so the costs of casting spells are two times higher for them than for liches. For example, if a spell costs a lich one unit of power, it would cost a human sorcerer 2 units of power. All spells in game are listed with their costs for liches.

If your phylactery is in your hand or on your person, you can expend 8 units of power to switch phylacteries. If you do so, your previous phylactery becomes a normal, non-magical object. Remove the card with your true name from inside the item’s envelope and destroy the card. Then take a card with a new true name from the ``Phylacteries'' sign and place it inside another non-magical item in your hand or on your person. This new item becomes your new phylactery. If you are Soul Bonded to someone, go out-of-game and immediately tell them what your new true name is.

Spells that can destroy a phylactery exist, though they are generally known only by magical creatures besides liches and humans. Phylacteries that are low on power are easier to destroy than ones containing a lot of power.

Corrupted phylacteries expend magic inefficiently, so liches must expend extra power to cast spells.

If your phylactery is destroyed, its power automatically reverts to you. You are now a human sorcerer with the same power level as before the phylactery's destruction. You no longer have a true name.

You can usually only add power to your phylactery by draining a dead human's lifeforce or by receiving a power transfer from another lich. Any two liches can instantaneously transfer power between themselves, provided both parties are willing and agree on the number of units to be transferred. Liches cannot transfer fractions of a unit.

Consuming a human's lifeforce typically bestows 1-3 units of power. Human sorcerers' lifeforces provide 3 units of power, regardless of how many units of power sorcerers had access to at the time of death. To consume a corpse's lifeforce, you must sit by the corpse and chant without interruption for one minute. At the end, the corpse disappears, and the power of their lifeforce is added to your phylactery. Ask a player out-of-game how much power you gain if you have drained their character's lifeforce.

If your phylactery is destroyed, making you a human once again, alert a GM.


\end{document}