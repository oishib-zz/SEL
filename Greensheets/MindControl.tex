\documentclass[green]{Sel}
\begin{document}
\name{\gMindControl{}}
If you have access to a lich's true name, you can issue magically binding orders that control their external actions, though not their internal thoughts or feelings. For example, you can order a lich to announce that they love oranges, but you cannot order them to truly change their opinion on oranges. If liches receive an order they may not be able to realistically accomplish, they must still do their best to fulfill it. To invoke a true name's power, you must precede every order by actually saying the name.

To use the true name's power, you must start off the order by actually calling the lich by their true name unless the lich is Soul Bonded to you. If you are ordering someone who is Soul Bonded to you, you can omit the name.

You must be within 1 ZOC to give magically binding orders. If you give an order within 1 ZOC and then either you or the lich move to be more than 1 ZOC apart, the order is no longer binding. Every order automatically stops being binding after five minutes.

If a lich receives competing orders, they must follow the most recent order.

Orders cannot be more than ten words long.

You can order a lich to act as if they are "willing" to perform or submit to mechanics. For example,  you can order a lich to act "willing" to be searched by you. You will then be able to search the lich, even if the lich is not internally content with being searched.

Only liches can be easily mind-controlled. Mind-controlling other creatures requires sophisticated preparation. If a lich has their phylactery destroyed, they can no longer be controlled through this mechanic.

Liches cannot mind control themselves.


\subsection{Soul Bonds}

A lich is usually subjected to mind control because someone else has accessed their phylactery. In rare cases, though, a lich falls in love with someone else, and the fragility of their souls can give rise to an extremely unusual type of magic known as a Soul Bond. According to common knowledge, Soul Bonds cannot be broken. Soul Bonded liches usually derive joy and fulfillment from their bonds, though a few are less content.

The moment a lich forms a Soul Bond, their beloved immediately learns their true name. If your character forms a Soul Bond, you should go out-of-character and tell your beloved's player your true name as soon as you can.

If a lich is Soul Bonded to you, orders you give them stay binding when you are within 3 ZOC of each other, not 1 ZOC.

If a lich is Soul Bonded to you, your orders are considered binding even if you do not initially say their true name.

If two liches are Soul Bonded to each other, they feel a psychological pull to comply with one another’s orders but can refuse to actually do so.


\end{document}
