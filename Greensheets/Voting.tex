\documentclass[green]{Sel}
\begin{document}
\name{\gVoting{}}
\textbf{Removal from the Society}

If anyone discovers that another member has violated any of the three core rules or engaged in behavior that is otherwise at odds with the Society's values, they are advised to declare their concerns publicly and move to remove the member from the Society. Any member can call a vote to remove a member from the Society at any time during a convention. If a vote is called, all members must gather together and openly place votes for or against removal by raising their hands. Every member must vote.

If no more than two of the Society's members vote against removal, the member is removed from the Society.

If the member who has been removed was previously the Leader, the current member who has belonged to the Society the longest automatically becomes Leader.

\textbf{Choosing a New Leader}

If the Society loses confidence in the current Leader, they can appoint a new one. Any member can nominate another member as a potential new Leader and call a vote at any time. A vote must proceed as above, and if no more than two members vote against appointing this member as a new Leader the vote passes.

If the new Leader was not already a Senior member, they automatically become one.

\textbf{Resolutions}

Immediately before the game begins, every member submitted one resolution, except for Domin, who submitted two. Members submit resolutions by writing them on a piece of paper and slipping them into a magic box. The box will remove any duplicates and compile the resolutions into a list, which members can read at the beginning of the game.

All members of the Society must take part in a vote, starting twenty minutes before the convention's end, to choose two new resolutions from the list posted at the start of the convention. The Leader must lead the voting process and ensure it occurs fairly and efficiently. To choose the first resolution, the voting process occurs as follows:

1.) Everyone openly votes for a resolution by raising their hand. Each member must vote for exactly one resolution.

2.) If one resolution wins more votes than all the others, it passes. Skip step 3.

3.) If multiple resolutions tie for having the same number of votes, go back to step 1 and hold a run-off vote between those resolutions. If two identical run-off votes occur (i.e. every member votes for the same resolution in one vote as they did in a prior vote) or four run-off votes occur, the Leader chooses the winning resolution out of the resolutions still in contention.

Remove the resolution that passes from the list. If the resolution that passes declares a research objective, remove all the other resolutions that would declare a research objective. Now choose the second resolution, repeating the process above.


\end{document}
